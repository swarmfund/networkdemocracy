%%%%%%%%%%%%%%%%%%%%%%%%%%%%%
%% Inline Symbols for CQM  %%
%%%%%%%%%%%%%%%%%%%%%%%%%%%%%



%% X structure / group structure
	\newcommand{\Xcolour}{Red}
	\newcommand{\XdotSym}{\hbox{\input{symbols/XdotSym.tex}}\!} % Dot
	\newcommand{\XmultSym}{\hbox{\input{symbols/timemultSym.tex}}\!} % Multiplication
	\newcommand{\XcomultSym}{\hbox{\begin{tikzpicture} [scale=1.2,transform shape] %% DO NOT CHANGE

\def\deltax{0.3} %% CAN BE CHANGED
\def\deltay{0.5} %% DO NOT CHANGE

\path[use as bounding box] (-\deltax,-\deltay) rectangle (\deltax,\deltay);

\node (mult_label_outl) at (-\deltax,+\deltay) {};
\node (mult_label_outr) at (+\deltax,+\deltay) {};
\node [dot, fill=\groupStructColour] (mult) at (0,0) {};
\node (mult_label_in) at (0,-\deltay) {};
\draw[-] [in=270,out=135] (mult) to (mult_label_outl);
\draw[-] [in=270,out=45] (mult) to (mult_label_outr);
\draw[-] (mult_label_in) to (mult);

%\draw (current bounding box.south west) rectangle (current bounding box.north east);
\end{tikzpicture}
}\!} % Comultiplication
	\newcommand{\XunitSym}{\hbox{\input{symbols/timeunitSym.tex}}\!} % Unit
	\newcommand{\XcounitSym}{\hbox{\input{symbols/timecounitSym.tex}}\!} % Counit
	\newcommand{\antipodeSym}{\hbox{\input{symbols/antipodeSym.tex}}\!} % Antipode (group inverse)
	\newcommand{\kpoints}[1]{K_{#1}} % Classical points of a given observable
	\newcommand{\Xkpoints}{\kpoints{\XdotSym}} % Classical points for the X observable


	
%% Z structure / point structure
	\newcommand{\Zcolour}{YellowGreen}
	\newcommand{\ZdotSym}{\hbox{\input{symbols/ZdotSym.tex}}\!} % Dot
	\newcommand{\ZmultSym}{\hbox{\input{symbols/timematchSym.tex}}\!} % Multiplication
	\newcommand{\ZcomultSym}{\hbox{\input{symbols/timediagSym.tex}}\!} % Comultiplication
	\newcommand{\ZunitSym}{\hbox{\input{symbols/timematchunitSym.tex}}\!} % Unit
	\newcommand{\ZcounitSym}{\hbox{\input{symbols/trivialcharSym.tex}}\!} % Counit
	\newcommand{\Zkpoints}{\kpoints{\ZdotSym}} % Classical points for the Z observable



%% Alternative X structure / alternative group structure (e.g. for quotients)
	\newcommand{\Xaltcolour}{Purple}
	\newcommand{\XaltdotSym}{\hbox{\input{symbols/XaltdotSym.tex}}\!}
	\newcommand{\XaltmultSym}{\hbox{\begin{tikzpicture} [scale=1,transform shape] %% DO NOT CHANGE

\def\deltax{0.3} %% CAN BE CHANGED
\def\deltay{0.5} %% DO NOT CHANGE

%\path[use as bounding box] (-\deltax,-\deltay) rectangle (\deltax,\deltay);

\node (mult_label_inl) at (-\deltax,-\deltay) {};
\node (mult_label_inr) at (+\deltax,-\deltay) {};
\node [dot, fill=\internalgroupStructColour] (mult) at (0,0) {};
\node (mult_label_out) at (0,+\deltay) {};

\draw[-] [out=90,in=225](mult_label_inl) to (mult);
\draw[-] [out=90,in=315](mult_label_inr) to (mult);
\draw[-] (mult) to (mult_label_out);

%\draw (current bounding box.south west) rectangle (current bounding box.north east);
\end{tikzpicture}
}\!} % Multiplication
	\newcommand{\XaltcomultSym}{\hbox{\input{symbols/internaltimecomultSym.tex}}\!} % Comultiplication
	\newcommand{\XaltunitSym}{\hbox{\input{symbols/internaltimeunitSym.tex}}\!} % Unit
	\newcommand{\XaltcounitSym}{\hbox{\input{symbols/internaltimecounitSym.tex}}\!} % Counit
	\newcommand{\Xaltkpoints}{\kpoints{\XaltdotSym}} % Classical points for the alternative X observable
	
	
	
%% Alternative Z structure / alternative point structure
	\newcommand{\Zaltcolour}{Cyan}
	\newcommand{\ZaltdotSym}{\hbox{\input{symbols/ZaltdotSym.tex}}\!}
	\newcommand{\ZaltmultSym}{\hbox{\input{symbols/internaltimematchSym.tex}}\!} % Multiplication
	\newcommand{\ZaltcomultSym}{\hbox{\input{symbols/internaltimediagSym.tex}}\!} % Comultiplication
	\newcommand{\ZaltunitSym}{\hbox{\input{symbols/internaltimematchunitSym.tex}}\!} % Unit
	\newcommand{\ZaltcounitSym}{\hbox{\input{symbols/internaltrivialcharSym.tex}}\!} % Counit
	\newcommand{\Zaltkpoints}{\kpoints{\ZaltdotSym}} % Classical points for the alternative Z observable



%% Discrete structure (aliases the black structure)
	\newcommand{\Dcolour}{black}
	\newcommand{\DdotSym}{\hbox{\input{symbols/DdotSym.tex}}\!} % Dot	
	\newcommand{\DcomultSym}{\hbox{\input{symbols/DcomultSym.tex}}\!} % Comultiplication	
	\newcommand{\DmultSym}{\hbox{\input{symbols/DmultSym.tex}}\!} % Multiplication
	\newcommand{\DcounitSym}{\hbox{\input{symbols/DcounitSym.tex}}\!} % Counit
	\newcommand{\DunitSym}{\hbox{\input{symbols/DunitSym.tex}}\!} % Unit



%% X structure in B/W (black structure)
	\newcommand{\Xbwcolour}{black}
	\newcommand{\XbwdotSym}{\hbox{\input{symbols/DdotSym.tex}}\!} % Dot	
	\newcommand{\XbwcomultSym}{\hbox{\input{symbols/DcomultSym.tex}}\!} % Comultiplication	
	\newcommand{\XbwmultSym}{\hbox{\input{symbols/DmultSym.tex}}\!} % Multiplication
	\newcommand{\XbwcounitSym}{\hbox{\input{symbols/DcounitSym.tex}}\!} % Counit
	\newcommand{\XbwunitSym}{\hbox{\input{symbols/DunitSym.tex}}\!} % Unit



%% Z structure in B/W (white structure)
	\newcommand{\Zbwcolour}{white}
	\newcommand{\ZbwdotSym}{\hbox{\input{symbols/ZbwdotSym.tex}}\!} % Dot	
	\newcommand{\ZbwcomultSym}{\hbox{\input{symbols/ZbwcomultSym.tex}}\!} % Comultiplication	
	\newcommand{\ZbwmultSym}{\hbox{\input{symbols/ZbwmultSym.tex}}\!} % Multiplication	
	\newcommand{\ZbwcounitSym}{\hbox{\begin{tikzpicture} [scale=0.8,transform shape] %% DO NOT CHANGE

\def\deltax{0.3} %% CAN BE CHANGED
\def\deltay{0.5} %% DO NOT CHANGE

\path[use as bounding box] (-\deltax,-\deltay) rectangle (\deltax,\deltay);

\node [dot, fill=\Zbwcolour] (mult) at (0,0.25*\deltay) {};
\node (mult_label_in) at (0,-\deltay) {};
\draw[-] (mult_label_in) to (mult);

%\draw (current bounding box.south west) rectangle (current bounding box.north east);
\end{tikzpicture}
}\!} % Counit
	\newcommand{\ZbwunitSym}{\hbox{\input{symbols/ZbwunitSym.tex}}\!} % Unit
	\newcommand{\ZbwleftDecohSym}{\hbox{\begin{tikzpicture} [scale=1,transform shape] %% DO NOT CHANGE

\def\deltax{0.3} %% CAN BE CHANGED
\def\deltay{0.5} %% DO NOT CHANGE

%\path[use as bounding box] (-\deltax,-\deltay) rectangle (\deltax,\deltay);

\node (mult_label_outl) at (-\deltax,+0.7*\deltay) [upground,scale = 0.3]{};
\node (mult_label_outr) at (+\deltax,+\deltay) {};
\node [dot, fill=\Zbwcolour] (mult) at (0,0) {};
\node (mult_label_in) at (0,-\deltay) {};
\draw[-] [in=270,out=135] (mult) to (mult_label_outl.180);
\draw[-] [in=270,out=45] (mult) to (mult_label_outr);
\draw[-] (mult_label_in) to (mult);

%\draw (current bounding box.south west) rectangle (current bounding box.north east);
\end{tikzpicture}}\!} % Left decoherence	
	\newcommand{\ZbwrightDecohSym}{\hbox{\input{symbols/ZbwrightDecohSym.tex}}\!} % Right decoherence	



%% Traces	
	\newcommand{\traceSym}{\hbox{\begin{tikzpicture} [scale=1.2,transform shape] %% DO NOT CHANGE

\def\deltax{0.3} %% CAN BE CHANGED
\def\deltay{0.5} %% DO NOT CHANGE

\path[use as bounding box] (-\deltax,-0.7*\deltay) rectangle (\deltax,0.3*\deltay);

\node (mult) at (0,0.3*\deltay) [upground,scale=0.5] {};
\node (mult_label_in) at (0,-0.7*\deltay) {};
\draw[-] (mult_label_in) to (mult);

%\draw (current bounding box.south west) rectangle (current bounding box.north east);
\end{tikzpicture}
}\!} % Trace symbol
	\newcommand{\trace}[1]{\traceSym_{#1}} % Trace
	\newcommand{\traceSymAlt}{\top} % Alternative trace symbol
	\newcommand{\traceAlt}[1]{\traceSymAlt_{#1}} % Alternative trace



%% Unlabelled symbols for controlled dynamics (algebras), projector-valued spectra (coalgebras), etx
	\newcommand{\algebraSym}{\hbox{\begin{tikzpicture} [scale=0.6,transform shape] %% DO NOT CHANGE


\path[use as bounding box] (-10mm,-10mm) rectangle (10mm,10mm);

\node [medium map dag, fill = gray] (algebra) at (0,0) {};

\node (H_in) [below of = algebra]{};
\node (H_out) [above of = algebra]{};
\node (G_in) [below of = algebra, xshift = 10mm]{};

\begin{pgfonlayer}{background}
\draw[-] [out=90,in=270] (H_in) to (algebra);
\draw[-] [out=90,in=270] (G_in) to (algebra.315);
\draw[-] [out=90,in=270] (algebra) to (H_out);
\end{pgfonlayer}


\end{tikzpicture}
}\!\!}	% Gray algebra symbol (2 inputs, 1 output)
	\newcommand{\measurementSym}{\hbox{\begin{tikzpicture} [scale=0.6,transform shape] %% DO NOT CHANGE


\path[use as bounding box] (-10mm,-10mm) rectangle (10mm,10mm);

\node [medium map, fill = gray] (algebra) at (0,0) {};

\node (H_in) [above of = algebra]{};
\node (H_out) [below of = algebra]{};
\node (G_in) [above of = algebra, xshift = 10mm]{};

\begin{pgfonlayer}{background}
\draw[-] [in=90,out=270] (H_in) to (algebra);
\draw[-] [in=90,out=270] (G_in) to (algebra.45);
\draw[-] [in=90,out=270] (algebra) to (H_out);
\end{pgfonlayer}


\end{tikzpicture}
}\!\!} % The adjoint of the algebra symbol
	\newcommand{\repSym}{\hbox{\begin{tikzpicture} [scale=0.6,transform shape] %% DO NOT CHANGE


\path[use as bounding box] (-10mm,-10mm) rectangle (10mm,10mm);

\node [medium map dag, fill = gray] (algebra) at (0,0) {};

\node (H_in) [above of = algebra, xshift = 2mm]{};
\node (H_out) [above of = algebra, xshift = -2mm]{};
\node (G_in) [below of = algebra]{};

\begin{pgfonlayer}{background}
\draw[-] [out=90,in=270] (algebra.60) to (H_in);
\draw[-] [out=90,in=270] (G_in) to (algebra);
\draw[-] [out=90,in=270] (algebra.120) to (H_out);
\end{pgfonlayer}


\end{tikzpicture}
}\!\!} % Internalised representation G -> H x H*
	\newcommand{\mapSym}{\hbox{\input{symbols/mapSym.tex}}\!\!} % The same as the measurement symbol, but with white bg.
	\newcommand{\mapconjSym}{\hbox{\input{symbols/mapconjSym.tex}}\!\!} % The conjugate of the map symbol.
	
	
%% Arrowed arcs - WARNING: can cause "size" issues at compile time if arcs too small or crowded
	\tikzset{->-/.style={decoration={markings,mark=at position #1 with {\arrow{>}}},postaction={decorate}}}
	\tikzset{-<-/.style={decoration={markings,mark=at position #1 with {\arrow{<}}},postaction={decorate}}}